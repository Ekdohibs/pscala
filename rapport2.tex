\documentclass[a4paper,10pt]{article}

\usepackage[francais]{babel} 
%\usepackage[T1]{fontenc}
\usepackage[utf8]{inputenc}


\usepackage{amsfonts}
\usepackage{amsmath}
\usepackage{amssymb}

%\usepackage{algorithmic}
%\usepackage{algorithm}
\usepackage{listings}
\usepackage{url}
\usepackage{graphicx}

\usepackage{listings}
\usepackage{xcolor}
%\lstset{upquote=false,
%        columns=flexible,
%        keepspaces=true,
%        breaklines,
%        breakindent=0pt}
%\lstset{basicstyle=\ttfamily\footnotesize,
%        commentstyle=\itshape\color{green},
%        keywordstyle=\bfseries\color{red},
%				stringstyle=\it\color{blue}}
%\lstset{language={Python}}
%\lstset{xleftmargin=2em,xrightmargin=2em,aboveskip=\topsep,belowskip=\topsep,
%        frame=single,rulecolor=\color{blue},backgroundcolor=\color{blue!5},frame=tb}

\textwidth 173mm \textheight 235mm \topmargin -50pt \oddsidemargin
-0.45cm \evensidemargin -0.45cm

%\newcommand{\code}[1]{#1}
\newcommand{\code}[1]{{\fontfamily{pcr}\selectfont #1}}

%\renewcommand{\thesection}{}
%\renewcommand{\thesubsection}{}

\title{Rapport - Compilation : Petit Scala, deuxième partie}
\author{Nathanaël Courant, Noémie Cartier}


\begin{document}

\maketitle

\section*{Choix techniques}

\subsection*{Production de code}

Nous avons choisi de compiler les expressions en plaçant leur résultat
dans \code{\%{}rax}~: ainsi, \code{compile\_{}expr} prend
en arguments une expression, la représentation des classes en mémoire
(voir plus loin), et le nombre d'arguments de la fonction actuelle, et
compile cette expression en écrivant son résultat dans
\code{\%{}rax}. \\ \par
Les expressions égales à \code{()} ne remettent pas \code{\%{}rax} à
$0$, la valeur de \code{()} n'étant jamais utilisée. En effet, les
règles de typage interdisent toute utilisation de la valeur d'un objet
de type \code{Unit}, celui-ci ne pouvant ni être converti en une
valeur d'un autre type, ni testé pour son égalité (avec un autre objet
de type \code{Unit}). \\ \par
Le valeur de retour de \code{compile\_{}expr} est une paire de type
\code{X86\_{}64.text * X86\_{}64.data}, représentant l'assembleur à
écrire dans le segment de texte et de données, respectivement. \\ \par
L'appel d'une fonction se fait en passant tous les paramètres de
celle-ci sur la pile, en poussant le premier paramètre (l'objet
lui-même) en premier. \\ \par
L'appel du constructeur d'une classe se déroule en commençant par
créer l'objet (i.e. l'allouer et remplir l'en-tête avec l'adresse du
descripteur de classe), puis écrire les paramètres du constructeur à
leur place respective dans l'objet, là où ils sont stockés plus
tard. On termine par appeler le constructeur comme une autre fonction.
Lorsque la classe a une classe parente, le constructeur calcule donc
les paramètres qui sont passés au parent et les stocke dans l'objet,
puis appelle le constructeur de la classe parente, et enfin calcule
les nouveaux champs qui sont définis dans cette classe. \\ \par
Ainsi, si classe B étend la classe A, la représentation mémoire de B
est la suivante~: \\ \par
\begin{center}
\begin{tabular}{|c|}
\hline
\code{descripteur de B} \\ \hline
\code{paramètres du constructeur de A} \\ \hline
\code{champs définis dans A} \\ \hline
\code{paramètres du constructeur de B} \\ \hline
\code{champs définis dans B} \\ \hline
\end{tabular}
\end{center}

Notre code initial évaluait les expressions binaires arithmétiques de
droite à gauche, pour simplifier le code produit. Cependant, nous
n'avons pas pu trouver d'informations sur l'ordre d'évaluation des
paramètres des opérateurs binaires en Scala, et les tests effectués
avec \code{scalac} semblaient indiquer que celui-ci était de gauche à
droite~; nous avons donc changé cela dans notre compilateur. \\ \par
Finalement, le calcul de la représentation mémoire des objets et des
descripteurs de classes se fait au tout début, dans l'ordre de
définition des classes. \\ \par
La fonction gérant la production de code n'a pas besoin de faire de
l'analyse de portée ou de détermination des méthodes appelées, ceci
étant déjà fait au cours du typage.

\end{document}