\documentclass[a4paper,10pt]{article}

\usepackage[francais]{babel} 
%\usepackage[T1]{fontenc}
\usepackage[utf8]{inputenc}


\usepackage{amsfonts}
\usepackage{amsmath}
\usepackage{amssymb}

%\usepackage{algorithmic}
%\usepackage{algorithm}
\usepackage{listings}
\usepackage{url}
\usepackage{graphicx}

\usepackage{listings}
\usepackage{xcolor}
%\lstset{upquote=false,
%        columns=flexible,
%        keepspaces=true,
%        breaklines,
%        breakindent=0pt}
%\lstset{basicstyle=\ttfamily\footnotesize,
%        commentstyle=\itshape\color{green},
%        keywordstyle=\bfseries\color{red},
%				stringstyle=\it\color{blue}}
%\lstset{language={Python}}
%\lstset{xleftmargin=2em,xrightmargin=2em,aboveskip=\topsep,belowskip=\topsep,
%        frame=single,rulecolor=\color{blue},backgroundcolor=\color{blue!5},frame=tb}

\textwidth 173mm \textheight 235mm \topmargin -50pt \oddsidemargin
-0.45cm \evensidemargin -0.45cm

%\newcommand{\code}[1]{#1}
\newcommand{\code}[1]{{\fontfamily{pcr}\selectfont #1}}

\title{Rapport - Compilation : Petit Scala, première partie}
\author{Nathanaël Courant, Noémie Cartier}

\begin{document}

\maketitle

\section{Choix techniques}

\subsection{Analyseur lexical}

Nous avons choisi, pour des raisons de performance et de minimisation
du nombre d'états de l'automate de l'analyseur lexical, de stocker les
mots-clef et le lexème correspondant dans une table de hachage. En
revanche, comme il est possible de placer plusieurs opérateurs à la
suite sans séparateur (mais que tous les opérateurs ne sont pas
réduits à un signe) nous avons réalisé leur reconnaissance en écrivant
un cas par opérateur.

\subsection{Analyseur syntaxique}


Nous avons beaucoup utilisé la bibliothèque de Menhir, 
en particulier les fonctions 
\code{option}, \code{delimited}, \code{list} et \code{separated\_{}list}. 
La fonction \code{ioption} nous a permis de résoudre un conflit.

Les constantes valant précisément $-2^{31}$ nous ont posé des
problèmes~: en effet, elles ne peuvent pas être reconnues
immédiatement par l'analyseur lexical, puisque celui-ci n'est pas
capable d'identifier si le symbole \code{-} devant la constante est
celui de la constante, ou est un opérateur binaire. De plus, nous
n'avons pas réussi à écrire une grammaire sans conflits
qui permettrait de détecter ces constantes négatives lors de l'analyse
syntaxique. Ce que nous avons dont fait a été de rajouter une phase de
traitement après l'analyse syntaxique, où les \code{-} unaires
précédant un entier sont transformés en constantes entières négatives,
et où les débordements de constantes sont détectés.

\subsection{Typeur}

Dans la définition d'un type (visible au début du fichier
\code{type\_{}ast.ml}), nous avons créé trois constructeurs différents
afin de mieux gérer la portée des différentes variables de type.

De la même façon, nous avons choisi d'utiliser une structure de type
\code{Map} au lieu de \code{Hashtbl} pour avoir des environnements
persistants.

Nous avons géré les paramètres de constructeur de classe comme des
champs privés, auquels on ne peut accéder que si l'expression de
l'objet est \code{this}. Ainsi, si \code{a} est un paramètre du
constructeur de la classe T, les expressions \code{a} et \code{this.a}
sont autorisées, tandis que \code{(new~T(0)).a} et
\code{\{();~this\}.a} ne le sont pas, comme en Scala. Une autre
possibilité aurait été d'accepter les accès de ce type uniquement
lorsque le type de l'expression de laquelle on essaie d'accéder au
paramètre du constructeur est exactement celle de la classe (et non
pas un sous-type de celle-ci), ce qui aurait préservé la sûreté du
typage, mais nous avons choisi de faire comme Scala pour ce point-là.

Dans le calcul de la variance, nous avons dû prendre un double
paramètre en permanence~: la classe (ou le type, selon les arguments
nécessaires pour chaque fonction) créée par le typeur et celle fournie
par l'analyseur syntaxique, les premiers portant les informations de
typage qui a déjà été fait (portée des variables de type, type des
champs lorsque celui-ci n'est pas précisé en particulier), et les
deuxième portant la localisation à afficher en cas d'erreur.

Nous décorons également l'arbre de syntaxe abstraite au cours du
typage, en ajoutant à chaque sous-expression son type. Au cours de
cette phase, on effectue également les opérations suivantes~:
\begin{itemize}
\item{détection de la portée des variables~: une variable peut soit
    être une variable locale, soit un paramètre de la méthode appelée,
  soit un paramètre du constructeur de la classe à l'intérieur de
  laquelle elle apparaît~;}
\item{détection de la méthode appelée~: lors d'un appel de méthode, on
    identifie de quelle classe cette méthode provient (i.e. dans
    quelle classe cette méthode a été définie en premier, en remontant
    dans la hiérarchie des classes tant que cette méthode existe),
    afin de pouvoir compiler ces appels ultérieurement.}
\end{itemize}

\textit{Modification -- 08/12/2015}~:
Nous avons remarqué que l'instanciation de paramètres de types, ou de
classes définies par défaut peuvent conduire à une perte de la sûreté
du système de typage~: en effet, cela permet en particulier une
création d'objets de type \code{Nothing}. On ne peut pas non plus
créer d'instance d'un paramètre de type, même si celui-ci est déclaré
comme héritant d'une autre classe, puisque les arguments du
constructeur peuvent différer (et même changer de nombre) entre une
classe et une classe dont elle hérite~! Par conséquent, nous avons
interdit toute instanciation d'un paramètre de type d'une classe,
ainsi que des classes de base (pour celles-ci, l'instantiation devant
soit être interdite, soit est inutile car correspond à créer des
objets de type \code{Unit}, \code{Int}, \code{String} ou
\code{Boolean}, pour lesquels il existe déjà d'autres manières de les
créer, soit de type \code{Any}, \code{AnyRef} ou \code{AnyVal}, qui
n'ont aucune utilité, car il n'est rien possible de faire avec ces
objets).


\section{Tests supplémentaires}

Nous nous aperçus sommes que, bien que notre compilateur passe déjà
tous les tests fournis pour le projet, il ne se comportait pas comme
il aurait dû dans certains cas particuliers. Nous avons donc créé
quelques nouveaux tests (ceux dont le nom est de la forme
\code{extra-*.scala}).

Ces tests supplémentaires vérifient ces cas particuliers~:
\begin{itemize}
\item{vérification du fait que la constante $-2^{31}$ est acceptée~;}
\item{vérification de la variance dans un champ d'une classe dont le
    type n'est pas précisé dans le programme~;}
\item{vérification de la bonne gestion des variables de type lorsque
    leur nom est commun à un paramètre de méthode et à un paramètre de
  classe, ou à une classe définie précédemment~;}
\item{vérification de la bonne gestion de variables de type nommées
    comme une classe standard (\code{Int}, \code{Unit}, ...).}
\item{\textit{Modification -- 08/12/2015}~: vérification qu'il est
    impossible d'instancier des paramètres de types ou des classes de
    base.}
\end{itemize}

\end{document}