\documentclass[a4paper,10pt]{article}

\usepackage[francais]{babel} 
%\usepackage[T1]{fontenc}
\usepackage[utf8]{inputenc}


\usepackage{amsfonts}
\usepackage{amsmath}
\usepackage{amssymb}

%\usepackage{algorithmic}
%\usepackage{algorithm}
\usepackage{listings}
\usepackage{url}
\usepackage{graphicx}

\usepackage{listings}
\usepackage{xcolor}
%\lstset{upquote=false,
%        columns=flexible,
%        keepspaces=true,
%        breaklines,
%        breakindent=0pt}
%\lstset{basicstyle=\ttfamily\footnotesize,
%        commentstyle=\itshape\color{green},
%        keywordstyle=\bfseries\color{red},
%				stringstyle=\it\color{blue}}
%\lstset{language={Python}}
%\lstset{xleftmargin=2em,xrightmargin=2em,aboveskip=\topsep,belowskip=\topsep,
%        frame=single,rulecolor=\color{blue},backgroundcolor=\color{blue!5},frame=tb}

\textwidth 173mm \textheight 235mm \topmargin -50pt \oddsidemargin
-0.45cm \evensidemargin -0.45cm

\newcommand{\code}[1]{#1} %{{\fontfamily{pcr}\selectfont #1}}

\title{Rapport - Compilation : Petit Scala, première partie}
\author{Nathanaël Courant, Noémie Cartier}

\begin{document}

\maketitle

\section{Choix techniques}

\subsection{Analyseur lexical}

Nous avons choisi, pour des raisons de performance et de minimisation du nombre 
d'états de l'automate de l'analyseur lexical, de stocker les mots-clef et le lexème 
correspondant dans une table de hachage. En revanche, comme il est possible de 
placer plusieurs opérateurs à la suite (mais que tous les opérateurs ne sont pas 
réduits à un signe) nous avons réalisé leur reconnaissance en écrivant un cas par 
opérateur.

\subsection{Analyseur syntaxique}

Nous avons choisi d'ignorer les constantes qui valent précisément $-2^{31}$, car 
les constantes sont reconnues par l'analyseur syntaxique avant de déterminer si 
elles sont précédées d'un moins unaire.

Nous avons beaucoup utilisé la bibliothèque de Menhir, 
en particulier les fonctions 
\code{option}, \code{delimited}, \code{list} et \code{separated\_{}list}. 
La fonction \code{ioption} nous a permis de résoudre un conflit.

\subsection{Typeur}

Dans la définition d'un type (visible au début du fichier \code{type\_{}ast.ml}), nous 
avons créé trois constructeurs différents afin de mieux gérer la portée des 
différentes variables de type.

De la même façon, nous avons choisi d'utiliser une structure de type \code{Map} au lieu de \code{Hashtbl} pour avoir des environnements persistants.

Dans le calcul de la variance, nous avons dû prendre un double paramètre en 
permanence~: la classe (ou le type, selon les arguments nécessaires pour chaque 
fonction) créée par le typeur et celle fournie par l'analyseur syntaxique, les 
premiers portant les information de typage qui a déjà été fait (portée des 
variables de type...), et les deuxième portant la localisation à afficher en cas 
d'erreur.

\section{Tests supplémentaires}

Nous nous sommes que, bien que notre typeur passe déjà tous les tests fournis pour 
le projet, il ne se comportait pas comme il aurait dû dans certains cas 
particuliers. Nous avons donc créé quelques nouveaux tests (ceux dont le nom est de 
la forme \code{extra*.scala}).

Par exemple, la vérification de la variance du type d'un champ de classe lorsque 
celui-ci n'est pas précisé dans le programme~: dans ce cas, le typeur ne vérifiait 
pas du tout la variance.

\end{document}